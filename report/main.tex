\documentclass[a4paper,11pt,headsepline]{article}

\usepackage[margin=1.75cm,includeheadfoot,footskip=30pt]{geometry}
%\usepackage{fullpage} % Use the full page
\usepackage{paralist}		% List environment
\usepackage{color}		% For colored text
\usepackage{times}
\usepackage{amsfonts}		% Additional math fonts
\usepackage{amsmath}		% Math symbols
\usepackage{latexsym}
\usepackage{graphicx}		% For including images
% \usepackage{listings}		% If listings are needed
\usepackage{graphicx,calc} % for pix
\usepackage{subfigure}
\usepackage{color}    % for coloured text
\usepackage{ifthen}
\usepackage{paralist}
\usepackage{times}
\usepackage{longtable}
\usepackage{colortbl}
\usepackage{nomencl}
% \usepackage{wrapfig}		% To wrap images
% \usepackage{algorithmic}	% Nice algorithm environment
% \usepackage{algorithm}
\usepackage{fancyhdr}		% Produce the nice header
% !Tex root = main.tex
\usepackage{acro} % acronyms

\DeclareAcronym{TAD}{
  short = TAD ,
  long  = Task-Aware-Downscaling,
  class = abbrev
}
\DeclareAcronym{LR}{
  short = LR,
  long  = low-resolution image,
  class = abbrev
}
\DeclareAcronym{SLR}{
  short = SLR,
  long  = task aware low-resolution image,
  class = abbrev
}
\DeclareAcronym{HR}{
  short = HR,
  long  = high-resolution image,
  class = abbrev
}
\DeclareAcronym{SHR}{
  short = SHR,
  long  = task aware-high-resolution image,
  class = abbrev
}
\DeclareAcronym{GRY}{
  short = GRY,
  long  = grayscale image,
  class = abbrev
}
\DeclareAcronym{SGRY}{
  short = SGRY,
  long  = task aware grayscale image,
  class = abbrev
}
\DeclareAcronym{COL}{
  short = COL,
  long  = colored image,
  class = abbrev
}
\DeclareAcronym{SCOLT}{
  short = COL,
  long  = task aware colored image,
  class = abbrev
}
\DeclareAcronym{SR}{
  short = SR,
  long  = Super-Resolution,
  class = abbrev
}
\DeclareAcronym{IC}{
  short = IC,
  long  = Image-Colorization,
  class = abbrev
}
\DeclareAcronym{SISR}{
  short = SISR,
  long  = Single-Image-Super-Resolution,
  class = abbrev
}
\DeclareAcronym{VSR}{
  short = VSR,
  long  = Video Super-Resolution,
  class = abbrev
}
\DeclareAcronym{PSNR}{
  short = PSNR,
  long  = Peak signal-to-noise ratio,
  class = abbrev
}


%% ----------------------------------------------------------------------------
% Definitions and appearance.
%% ----------------------------------------------------------------------------
\renewcommand{\vec}[1]{\mathbf{#1}}
\newcommand{\mat}[1]{\mathtt{#1}}
\newcommand{\pinv}{^{+}}
\newcommand{\inv}{^{-1}}
\newcommand{\trans}{^{\!\top}}
\newcommand{\invtrans}{^{-\!\top}}
\newcommand{\myref}[1]{(\ref{#1})}
\newcommand{\myeqref}[1]{Eq.~\myref{#1}}
\newcommand{\myfigref}[1]{Fig.~\ref{#1}}
\newcommand{\mychapterref}[1]{Chapter~\ref{#1}}
\newcommand{\mysecref}[1]{Section~\ref{#1}}
\newcommand{\myalgoref}[1]{Algorithm~\ref{#1}}

% Change the appearance of the header. Here \MakeUppercase is hard-coded,
% so renewing this command allows to elegantly change the header appearance.
\renewcommand{\MakeUppercase}{\scshape}

% Set the headings' appearance in the ``fancy'' pagestyle
\pagestyle{fancy}
\fancyhf{}
\rhead{Simon Schaefer}
\lhead{Efficient Task Aware Super-Resolution and Colorization}
\rfoot{\thepage}

\begin{document}
%% ----------------------------------------------------------------------------
% Preface of document.
%% ----------------------------------------------------------------------------
\pagenumbering{Roman}
\pagestyle{empty} % even no page number
\fancypagestyle{plain}{
  \renewcommand{\headrulewidth}{0.0pt}
  \fancyfoot{}
  \fancyhead{}
}
% Title page.
\begin{titlepage}

\thispagestyle{empty}

\includegraphics[height=2.5cm]{figures/ethlogo_black}
\vspace*{-0.2cm}\includegraphics[height=2.5cm]{figures/cvl}

\vspace*{2cm}
\begin{center}
\Huge{\textbf{Efficient Task Aware Super-Resolution and Colorization}\\}
\LARGE{\textbf{For Image and Video Domain}\\[1cm]}

\large{Semester Project\\[0.8cm]}
\LARGE{Simon Schaefer\\}
\end{center}

\vfill
\begin{center}
\begin{tabular}{ll}
\Large{\textbf Advisor:} & \Large{Dr. Radu Timofte, Shuhang Gu}\\
\Large{\textbf Supervisor:} & \Large{Prof.~Dr.~Luc van Gool}\\
			    & \small{Computer Vision Laboratory, ITET ETH}\\
\end{tabular}
\end{center}

\begin{center}
\today\\
\end{center}

\end{titlepage}

% Abstact.
% !Tex root = main.tex
\newpage
\section*{Abstract}
\addcontentsline{toc}{section}{Abstract}
\noindent The abstract gives a concise overview of the work you have done. The reader shall be able to decide whether the work which has been done is interesting for him by reading the abstract. Provide a brief account on the following questions:

\begin{itemize}
 \item What is the problem you worked on? (Introduction)
 \item How did you tackle the problem? (Materials and Methods)
 \item What were your results and findings? (Results)
 \item Why are your findings significant? (Conclusion)
\end{itemize}

\noindent The abstract should approximately cover half of a page, and does generally not contain citations.

% Input here any acknowledgements
% !Tex root = main.tex
\newpage
\section*{Acknowledgements}
%\addcontentsline{toc}{section}{Acknowledgements}

I would like to thank a number of people who have encouraged and helped us in
writing this Semester Project. I am proud of the achieved results and I
appreciate the support I received from all sides.
\vspace{0.5cm}
\newline
I am much obliged to Prof. Dr. Luc Van Gool for his support, for the confidence
in my project and for providing the opportunity to execute it.
\vspace{0.5cm}
\newline
Special thanks to my advisors Radu Timofte and Shuhang Gu. Their broad knowledge
and experiences in machine learning had a wide influence on the results and
also on my education.
\vspace{0.5cm}
\newline
Most importantly, I am very grateful to all my friends and family members
who supported us during the whole project and helped me to achieve my aims.
\vspace{0.5cm}
\newline
Zürich in June 2019,
\newline
Simon Schaefer

% % Chapter-pages etc. use the ``plain'' pagestyle - since we don't want to have
% a heading at all at chapter-pages, redefine plain accordingly.
\fancypagestyle{plain}{
  \renewcommand{\headrulewidth}{0.0pt}
  \fancyfoot{}
  \fancyfoot[RO, LE]{\thepage}
  \fancyhead{}
}

\pagestyle{fancy}

% Table of contents
\newpage
\tableofcontents
% List of figures
\newpage
\listoffigures
\addcontentsline{toc}{section}{List of Figures}
% List of tables
\listoftables
\addcontentsline{toc}{section}{List of Tables}
% Insert list of acronyms
\newpage
\addcontentsline{toc}{section}{Abbreviations}
\printacronyms[include-classes=abbrev,name=Abbreviations]
\newpage
\pagenumbering{arabic}

%% ----------------------------------------------------------------------------
% Actual text comes here - defer it to other files and use \input{bla.tex}, ..
%% ----------------------------------------------------------------------------
\chapter{Introduction}
Give an introduction to the topic you have worked on:

\begin{itemize}
 \item \textit{What is the rationale for your work?} Give a sufficient description of the problem, e.g. with a general description of the problem setting, narrowing down to the particular problem you have been working on in your thesis. Allow the reader to understand the problem setting.
 \item \textit{What is the scope of your work?} Given the above background, state briefly the focus of the work, what and how you did.
 \item \textit{How is your thesis organized?} It helps the reader to pick the interesting points by providing a small text or graph which outlines the organization of the thesis. The structure given in this document shows how the general structuring shall look like. However, you may fuse chapters or change their names according to the requirements of your thesis.
\end{itemize}


%\section{Focus of this Work}

%\section{Thesis Organization}

\newpage
\section{Related Work}
\label{sec:RelatedWork}

\subsection{Super-Resolution in Image Domain}
The problem of SR in the image domain is called \ac{SISR} and is shown in
\myfigref{fig:sisr_problem}. A lot of approaches have been
tried in order to cope with the \ac{SISR} problem. While early approaches such as
bicubic and Lanczos \cite{LFIOATD} tackle the problem using simple deterministic
filters which are computational cheap but produce blurry results and lack in
high frequency details, more recent approaches approach the problem using
example-based methods such as sparse encoding or deep learning methods.

\begin{figure}[!htbp]
	\centering
	\includegraphics[width=10cm]{figures/sisr_problem}
	\caption{General SISR problem  according to \cite{DLFSISRABR}.}
  \label{fig:sisr_problem}
\end{figure}

Sparsity-based techniques assumes the \ac{LR} image to be transformable in another
domain (usually a dictionary of image atoms \cite{SARPFTTAISAIP}) and tries to
find correspondences between the \ac{LR} and \ac{HR} patches in the transformed space, as
implemented in \cite{IDASRBASDSAAR}. However, these techniques usually are
very computationally expensive. Among other learning based approaches such as
the use of random forests \cite{FAAIUWSRF}, in-place example regression models
\cite{FISRBOIPER} or adjusted anchored neighborhood regression \cite{AANRFFSR},
in terms of accuracy applying CNN based approaches have shown the largest success.
\footnote{An overview of various other deep learning based approaches for SISR
can be found in \cite{DLFSISRABR}.}
Dong et al. \cite{LADCNFISR} trained a shallow CNN end-to-end to build the HR
image based on a bicubicly upscaled LR image. This approach was improved by Kim
et al. \cite{AISRUVDCN} (VDSR) using a deeper network (20 layers) and cascading
small filters many times in a deep network structure to exploit contextual
information over large image regions in an efficient way. By advancing the
network model VDSR was further improved by Lim et al. \cite{EDRNFSISR}.

\begin{figure}[!htbp]
	\centering
	\includegraphics[width=10cm]{figures/vdsr}
	\caption{Overview of VDSR network design \cite{AISRUVDCN}.}
  \label{fig:vdsr}
\end{figure}

\subsection{Super-Resolution in Video Domain}
\ac{VSR} combines information from multiple adjacent LR frames
to take temporal information into account, leading to higher quality results.
Takeda et al. \cite{SRWESME} apply a 3D kernel regression on a patch of adjacent
\ac{LR} frames to implicitly encounter temporal information. Since purposed by
Caballero et al. \cite{RTVSRWSTNAMC} end-to-end approaches including motion
compensation such as the CNN framework from \cite{RTVSRWSTNAMC} have large success
in the VSR area. Liu et al. \cite{RVSRWLTD} added temporal addaptivity to the
framework to be able to aggregate the resulting \ac{HR} frame based on a weighted
sum of several estimates as well as a varying number of input LR frames. Sajjadi
et al. \cite{FRVSR} purposed a frame-recurrent architecture iteratively using
the previously inferred \ac{HR} frames for the subsequent prediction. Wang et al.
\cite{LFVSRTHROFE} (SOFVSR) implemented an end-to-end trainable approach to predict
both, the \ac{HR} frame as well as the HR optical flow. Therefore, first the HR
optical flow is inferred in a coarse-to-fine manner, then motion compensation is
performed according to the HR optical flows and finally, the compensated LR
inputs are fed to a super-resolution network to generate the HR frame estimate
(comp. \myfigref{fig:sofvsr}).\footnote{Since SOFVSR was used in the project
an overview of SOFVSR baselines can be found in the appendix.}

\subsection{Colorization}
Image colorization methods can be categorized in two categories: Non-parametric
approaches, such as \cite{ICUSI}, model the correspondence between the grayscale
and the colored image by finding analogeous regions in reference image(s),
while parameteric models learns this correspondence from large datasets,
transforming the colorization problem into a regression problem. Zhang et al.
\cite{CIC} (CIC) purpose posing colorization as a classification task and use
class-rebalancing at training time to increase the diversity of colors in the
result, not requiring any user-interaction.

\begin{figure}[!htbp]
	\centering
	\includegraphics[width=10cm]{figures/cic}
	\caption{Overview of CIC network design \cite{CIC}.}
  \label{fig:cic}
\end{figure}

\subsection{Task-Aware-Downscaling}
Over all of the problems stated above most of the approaches merely take into
account one side of the process, e.g. by fixing the transformation HR to LR
to bicubic interpolation in order to large amount of training data and focusing
on estimating the inverse transformation. Kim et al. \cite{TAID} (TAID) purpose taking
into account the downscaling method in order to improve the upscaling performance,
by training an autoencoder in an end-to-end manner while the latent space
representation again is an image of same size as the LR image. The loss function
thereby contains both the difference between the decoded SHR and the original HR
image as well as the difference between the encoded SLR and the bicubic
interpolated LR image, such that the SLR image is a humanly understandable
representation. Next to SISR the approach is shown to be applicable for large
scale factor up to 128 as well as for colorization.

\begin{figure}[!htbp]
	\centering
	\includegraphics[width=8cm]{figures/taid_colorization}
	\caption{Qualitative results of TAID colorization for Set14 dataset.}
  \label{fig:taid_colorization}
\end{figure}

% Describe the other's work in the field, with the following purposes in mind:
%
% \begin{itemize}
%  \item \textit{Is the overview concise?} Give an overview of the most relevant work to the needed extent. Make sure the reader can understand your work without referring to other literature.
%  \item \textit{Does the compilation of work help to define the ``niche'' you are working in?} Another purpose of this section is to lay the groundwork for showing that you did significant work. The selection and presentation of the related work should enable you to name the implications, differences and similarities sufficiently in the ``discussion'' section.
% \end{itemize}

\newpage
\chapter{Materials and Methods}
\label{sec:Approach}

\section{General Problem Formulation}
\label{sec:Approach_GPF}
The general idea behind \ac{TAD} is that a high-dimensional input (e.g. a
high-resoluted or colored image) is transformed in a low-dimensional space so that it first can be reconstructed as good as possible and second still is human-understandable in lower dimensional space. Besides, both transformations should be computationally efficient, so that an optimal trade-off between network complexity (efficiency) and reconstruction capabilities is met.

\begin{figure}[!htbp]
	\centering
	\includegraphics[width=10cm]{figures/problem}
	\caption{General \ac{TAD} problem formulation.}
  \label{fig:problem}
\end{figure}

With $g_\phi$ the downscaling and $f_\theta$ the upscaling function, $X_{GT}$ the groundtruth (input) as well as $X_{SLR}$, $X_{SHR}$ its low- and high-dimensional representation, the \ac{TAD} problem can be formulated as combined optimization problem constraining both the low-dimensional representation (readability) as well as the high-dimensional reconstruction (accuracy). While the second constraint can be easily formulated using the input image $X_{GT}$ as groundtruth the first constraint is more vague and hard to quantify. Therefore, it is assumed that the optimal latent space encoding is similar to a trivially obtained low-dimensional representation like a (bilinearly) interpolated or greyscale image. As further described in \mysecref{sec:Approach_TS} $X_{SLR}$ is thereby not derived from scratch but builds up on the guidance image in the training procedure so that both optimization problems can be solved more independently than learning both $X_{SLR}$ and $X_{SHR}$ from scratch and typically the first optimization problem (readability of $X_{SLR}$) is easier to solve for the model.

\section{Autoencoder Network Architecture}
\label{sec:Approach_ANA}
As no groundtruth for the low resolution image is available, since \ac{TAD} poses requirements for both the down- and upscaling and because it has proven to work for the \ac{TAD} problem in previous works an autoencoder design is used.

\begin{figure}[!htbp]
	\centering
	\includegraphics[width=14cm]{figures/architecture_example.png}
	\caption{Example architecture of the \ac{TAD} autoencoder network design for \ac{SISR} task.}
  \label{fig:architecture}
\end{figure}

The autoencoder should be able to handle an input image of general size, it should be runtime-efficient, store as much information as possible while downscaling as well as end-to-end and efficiently trainable.
Therefore a convolutional-only, reasonable shallow network design is used. To avoid the loss of information during downscaling instead of pooling operations subpixel convolutional layers are employed. Furthermore, in order to enable efficient training and circumvent vanishing gradient problems (especially for larger networks that were tested) next to direct forward passes ResNet (\cite{DRLFIR}) like \textit{Resblocks} are used, which are structured as

$$Resblock(x) = x + Conv2D(ReLU(Conv2D(x)))$$

Since this network design does not continuously downscale the input but
applies pixel shuffling to downscale while all other layers do not alter their inputs shape, the networks also is easily adaptable to design changes, which simplifies the architecture optimization process.

\section{Loss Function}
\label{sec:Approach_LF}
The loss function $L$ consists of two parts, representing both optimization problems introduced in \mysecref{sec:Approach_GPF}. The first one, $L_{TASK}$, is task-dependent and states the difference between the decoders output $X_{SHR}$ and the desired output $X_{GT}$, e.g. the original \ac{HR} in the \ac{SISR} task.

$$L_{TASK} = L1(X_{GT}, X_{SHR})$$

The second part, $L_{LATENT}$, encodes the human-readability of the low-dimensional representation. So $L_{LATENT}$ is the distance between the interpolated guidance image $X_{GD}$ and the actual encoding $X_{SLR}$:

$$L_{LATENT} = \begin{cases}
L1(X_{GD}, X_{SLR}) & \text{if } ||L1/d_{max}|| \geq \epsilon
\\ 0.0 & \text{otherwise}
\end{cases}$$

with $||L1/d_{max}||$ being the $L1(X_{GD}, X_{SLR})$ loss normalized
by the maximal deviation between $X_{GD}$ and $X_{SLR}$. Hence, $L_{LATENT}$ is zero in an $\epsilon$-ball around the guidance image, ensuring that \ac{SLR} is close to the guidance image but also helps to prevent overfitting to the trivial solution $X_{GD} = X_{SLR} \Leftrightarrow g_\phi = 0$. As shown in \mychapterref{sec:ExperimentsandResults} introducing an $\epsilon$-ball also improves the model's robustness against perturbations.
The overall loss function is a weighted sum of both of the loss function introduced above. The relative weight $(\alpha, \beta)$ is of large importance for the trade-off between the readability requirement and the performance of the model's upscaling part (super-resolution, colorization). However, as described above the readability requirement is less strict so that typically $\alpha >> \beta$.

$$L = \alpha L_{TASK} + \beta L_{LATENT}$$

\section{Training Specifications}
\label{sec:Approach_TS}
Even if a guidance image is part of the loss function learning both the low- and high-dimensional representation from scratch poses a combined optimization problem which usually is very hard to solve. To ensure (faster) convergence, therefore in the beginning of the training procedure the guidance image is added to the encoder's output and the encoding is not discretized (finetuning). This improves both the convergence rate of $X_{SLR}$ and $X_{SHR}$, especially in the beginning of the training procedure, since merely a difference between the interpolated representation and the more optimal encoding has the be derived and the down- and upscaling can be learned more independently since the lower dimensional representation is always guaranteed to be useful for upscaling.

\begin{figure}[!htbp]
    \centering
    \includegraphics[height=5cm]{figures/guidance_loss.png}
    \includegraphics[height=5cm]{figures/guidance_psnrs.png}
    \caption{Loss curve without adding guidance image (left) and with
    adding guidance image (right) while training for \ac{IC} example.}
    \label{fig:loss_w_wo_adding_guidance}
\end{figure}

\begin{figure}[!htbp]
    \centering
    \includegraphics[height=5cm]{figures/finetuning_loss.png}
    \includegraphics[height=5cm]{figures/finetuning_psnrs_16.png}
    \caption{Loss curve without finetuning (left) and with finetuning (right) while training for \ac{SISR} (scale = $16$) example.}
    \label{fig:loss_w_wo_finetuning}
\end{figure}

\section{Task-Specific Design}
\label{sec:Approach_TSD}
While the general approach derived in \mysecref{sec:Approach_GPF} can be trivially applied on the \ac{SISR} and on the \ac{IC} problem the extension to the \ac{VSR} task is more advanced. In the following the specification of all three problems are presented:

\subsection*{Single-Image-Super-Resolution Problem}
An example of the overall design of the \ac{TAD} pipeline applied on the
\ac{SISR} problem is shown in \myfigref{fig:architecture}. The high-dimensional space here is a high-resolution image ($X_{GT} = X_{HR}$), while the low-resolution guidance image is the bilinearly downscaled image ($X_{GD} = X_{LR}^{B}$). As shown for an example of scaling factor 4 in \mysecref{sec:Experiments_SISR} the model can be trained more efficiently as well as performs better when iteratively scaling (i.e. scaling two times with factor 2, instead of one time with factor 4 directly).

\subsection*{Video-Super-Resolution}
As pointed out in \mychapterref{sec:RelatedWork} the challenge of \ac{VSR} compared to \ac{SISR} is to not only take one frame into account but subsequent frames in order to reconstruct opaqued objects, reflect motions, etc. As shown in \mychapterref{sec:RelatedWork} most approaches tackling the \ac{VSR} problem follow a two-step approach, firstly reconstructing the optical flow using the current and previous low-resolution, using the result to warp the image and finally upscale it. An improvement in any of these building blocks would improve the overall reconstruction, therefore there are multiple possibilities for finding a more optimal low-dimensional representation. Several approaches were tried including a direct approach which encodes the reconstruction capability of a given \ac{VSR} model into the loss function and tries to shape the low-dimensional representation such that the model's performance improves as well as an approach directly targeting the optical flow calculation (more details in the appendix).

\begin{figure}[!htbp]
	\centering
	\includegraphics[width=14cm]{figures/architecture_video_external}
	\caption{Design for \ac{TAD} video super-resolution task (example network
  architecture).}
  \label{fig:architecture_video}
\end{figure}

The most promising approach is displayed in \myfigref{fig:architecture_video} and involves a training in multiple steps. In the first step, the autoencoder model is trained on incoherent images (similar to \ac{SISR}) to learn a good general down- and upscaling transformation. In a second step the pretrained model is explicitly trained on a video dataset, in order to produce the training dataset for a \ac{VSR} model, which is trained on this SLR frames specifically in the third step. After training to validate the model subsequent frames are downscaled first using the video pretrained \ac{TAD} model, the downscaled images are then fed to the trained \ac{VSR} model, upscaling them. While this approach basically is applicable to all \ac{VSR} frameworks in the scope of this project the SOFVSR (\cite{LFVSRTHROFE}) model was used. \footnote{Further details about the selection of SOFVSR and a description of the approach itself can be found in the appendix.}

\subsection*{Image Colorization}
While in a grayscale image information about intensity, color value and
saturation are mingled over all channels, other color spaces split these
information in separate channels. In order to contain as much information about the original colors a non-uniformly-weighted (while grayscale would be a uniformly weighted sum, destroying color contrast information) and static (i.e. non periodic like hue in HSV color space) sum of original color values would be optimal, which is e.g. the Y channel of the YCbCr channel. Therefore it is used as guidance image ($X_{GD} = X_{GRY}^Y$), while the original colored image can be used as groundtruth ($X_{GT} = X_{COL}$).

\begin{figure}[!htbp]
    \centering
    \includegraphics[width=14cm]{figures/architecture_color_large.png}
    \caption{Design for \ac{TAD} colorization task (example network architecture).}
    \label{fig:architecture_color}
\end{figure}

\section{Implementation}
\label{sec:Approach_IMP}
The project was implemented in Python 3, using the PyTorch deep learning
framework. Although some ideas from Kim et al. \cite{TAID} were adopted as described above the pipeline had to be re-implemented from scratch and
re-validated since neither code nor any pretrained model have been
available publicly (nor upon request). As PyTorch merely supports
subpixel convolutional layers, their inverse transformation was implemented as well. Since most of the open-source models for the \ac{VSR} problem merely are available in Tensorflow, as is the used model, it was re-implemented and adapted to the \ac{TAD} pipeline.
\newline
During program development it was paid attention to generality and
commutability in order to efficiently test a variety of different models and datasets as well as guarantee comparability of different approaches. The full code stack can be found on \url{https://github.com/simon-schaefer/tar}.

% The objectives of the ``Materials and Methods'' section are the following:
% \begin{itemize}
%  \item \textit{What are tools and methods you used?} Introduce the environment, in which your work has taken place - this can be a software package, a device or a system description. Make sure sufficiently detailed descriptions of the algorithms and concepts (e.g. math) you used shall be placed here.
%  \item \textit{What is your work?} Describe (perhaps in a separate section) the key component of your work, e.g. an algorithm or software framework you have developed.
% \end{itemize}

% !Tex root = main.tex
\newpage
\section{Experiments and Results}
\label{sec:ExperimentsandResults}
In order to test the previously described \ac{TAD} approach several experiments
were performed, to find the optimal model for both the \ac{SISR} and \ac{IC}
tasks, to show the impact of the L1 ball on the model's robustness against
perturbations as well as evincing the feasibility of applying the \ac{TAD}
method in the video domain.

\begin{figure}[!htbp]
	\centering
	\includegraphics[width=14cm]{figures/model_adaptions}
	\caption{Model and Training adaptions during experiments in comparison
  to the baseline model by \cite{TAID}.}
  \label{fig:model_adaptions}
\end{figure}

As shown in \myfigref{fig:model_adaptions} a bunch of adjustments to the
baseline model (by \cite{TAID}) were tried for analysing the coherence between
the model complexity and its reconstruction performance. Thereby very small
architectures (6 layers) as well as comparable large architectures
(19 layers) were tested (baseline model has 10 layers), spanning
from $375.926$ to $1.299.126$ parameters. Since the model already is quite
shallow the removal of each layer had an impact on the resulting performance,
therefore especially the number of \textit{Resblocks} (two convolutional layers
and ReLU) has a huge impact on the reconstruction accuracy. Next to adaptions
to the baseline architecture completly new architectures have been tested, such
as networks without any residual pass (so no Resblocks) but convolutional layers
with either constant or varying (first increasing then decreasing) number of
filters (up to 256) have been used.
\newline
Next to several architectures the baseline model was improved by advancing the
training procedure. Next to enhancing the loss function as discussed in
\mysecref{sec:Approach_LF} instead of a constant an linearly annealing learning
rate was used, starting at $4*10^{-4}$ and annealing by factor $\gamma = 0.25$
after $20$, $100$ and again after $200$ training epochs. Adam optimizer was used
with $\beta = (0.9, 0.999)$, $\epsilon_{ADAM} = 10^{-8}$, gradient clipping and
zero weight decay.
\newline
To guarantee comparability to other super-resolution and colorization paper
in the image domain the model was trained using the DIV2K training dataset
(\cite{DIV2K}), while validated on the SET5 (\cite{SET5}), SET14 (\cite{SET14}),
BSDS100 (\cite{BSDS100}), URBAN100 (\cite{URBAN100}) and VDIV2K (\cite{DIV2K})
dataset. For similar reason for the video domain the model was pretrained using
DIV2K, actually trained on video clips from CDVL Database (Ntiaspen) and
validated on the widely known Vid4 dataset (Calendar, Foliage, Walk, City).
For improving generalization capabilities of the model and avoid overfitting
the image training data were also augmented (rotated, mirrored).
\newline
Similiarly, as widely used in the field of image reconstruction as a performance
measurement the \ac{PSNR} will be used.
\newline
A complete list of the most important testing configurations and their results
as well as a list of architectures can be found in the appendix.

\subsection{Impact of L1 Ball}
\label{sec:Experiments_EPS_BALL}
As already seen in \mychapterref{sec:Approach} introducing an $\epsilon$-ball
to the loss term pervents the model from overfitting on the low-dimensional
image, $X_{SLR} = X_{GD}$ (i.e. trivial solution $g_\theta = 0$ in the
beginning of the training).\footnote{As discussed in \mychapterref{sec:Approach}
the training is splitted into two parts, first learning only the difference
between the guidance image and a more optimal representation and later
learning the low-dimensional representation independent from the guidance
image.} However, the main contribution of the $\epsilon$-ball consists in the
increasing robustness against perturbations of $X_{SLR}$, which are modeled
as white Gaussian noise with standard deviation $\sigma$ within this project.
While a model trained with $\epsilon = 0$ is highly vulnerable to perturbations,
dropping \ac{PSNR} by $42 \%$ by adding noise with $\sigma = 0.11$ ($X_{SLR} \in
[0,1]^n$), a model trained with $\epsilon = 10$ is more stable dropping only
about $10 \%$ in the same scenario (scale = 4, dataset = SET14).

\begin{figure}[!htbp]
	\centering
	\includegraphics[width=8cm]{figures/epsball_qualitative_comp}
	\caption{Qualitative comparison between the impact of perturbation on a
  model trained without and with $\epsilon$-ball (scale = 4, dataset = SET14).}
  \label{fig:epsball_qualitative_comp}
\end{figure}

In the following experiment the same model (AETAD)\footnote{An overview over
all model architectures can be found in the appendix.} was trained with different
radii of the $\epsilon$-ball. It turns out that the right choice of $\epsilon$
is a trade-off between the reconstruction performance and robustness aginst
perturbations, comp. \myfigref{fig:epsball_perturbation}.

\begin{figure}[!htbp]
	\centering
	\includegraphics[width=10cm]{figures/epsball_perturbation}
	\caption{Reconstruction performance over several values of $\epsilon$
  and $\sigma$ (scale = 4).}
  \label{fig:epsball_perturbation}
\end{figure}

Also an increasing
$\epsilon$ improves the convergence rate during training, as shown in
\myfigref{fig:epsball_loss}, the model otherwise first overfits to $X_{GD}$
and then eventually finds a more optimal trade-off between fitting the
low- and high-dimensional image (with increasing $\frac{\alpha}{\beta}$ ratio).
However, as \myfigref{fig:epsball_perturbation} the radius of the $\epsilon$-ball
around $X_{GD}$ cannot be choosen infinitely large, since the overall performance
worsens as the impact of the guidance image on the model convergence decreases
(e.g. $\epsilon = 100$ in \myfigref{fig:epsball_perturbation}).

\begin{figure}[!htbp]
	\centering
	\includegraphics[width=6cm]{figures/epsball_loss_1_set14}
  \includegraphics[width=6cm]{figures/epsball_loss_50_set14}
	\caption{Closeness between $X_{SLR}$ and guidance image (blue) and $X_{SHR}$
  and groudtruth image (orange) for different values of $\epsilon$ (left:
  $\epsilon = 1$, right $\epsilon = 50$).}
  \label{fig:epsball_loss}
\end{figure}

\mytableref{table:epsilonotherscales} displays the impact of models trained
with different values of $\epsilon$ on the reconstruction capabilities on
non-trained scales. In general the model overfits on a scale it is trained on,
however allowing to deviate from the guidance image by increasing $\epsilon$
amplifies the effect of overfitting the learnt transformation to a the scaling
factor the model is trained on. Hence, as shown below especially for large
deviations in scaling factor the accuracy worsens with increasing $\epsilon$.

\begin{table}[!htbp]
	\begin{center}
	\begin{tabular}{c|c|c|c|c}
	epsilon & x2 & x4 & x8 & x16 \\
	\hline
	0 & 8.446 & 24.406 & 8.555 & 18.938 \\
	20 & 6.244 & 22.188 & 6.492 & 17.308 \\
	50 & 6.771 & 22.459 & 6.925 & 16.992 \\
	100 & 10.901 & 23.784 & 11.870 & 13.225 \\
	\end{tabular}
	\caption{Impact of $\epsilon$ on performance on non-trained scales
	(trained scale = 4, dataset = SET14).}
	\label{table:epsilonotherscales}
	\end{center}
\end{table}

Overall the choice of $\epsilon$ depends very much on the application and its
external conditions that should be solved, e.g. whether perturbations are
probable for example during the storing, downloading etc. process.
In case only the performance without any disturbances matters, $\epsilon < 10$
is a good choice, as it fastens convergence of the model, prevents overfitting
on the guidance image but also does not affect the accuracy without any
disturbances much. For the further experiments a value $\epsilon = 1$ was used.
\newline
Although described for the \ac{SISR} problem here the described impact of
the $\epsilon$-ball is similar over all other problems, which omitted here
for the matter of compactness of this report.

\subsection{Single-Image Super-Resolution}
\label{sec:Experiments_SISR}
Next to improving the robustness of the \ac{TAD} model several improvements
were made on both the way of training as well as on the architecture itself.
\myfigref{fig:psnr_complexty_sisr} shows the correlation on both the SET5
and the SET14 dataset for different architectures which were trained using the
same parameters (learning parameters as described above, $\epsilon = 10$) for
the sake of a comparability.

\begin{figure}[!htbp]
	\centering
	\includegraphics[width=18cm]{figures/psnr_complexity_sisr}
	\caption{Model complexity vs reconstruction performance for \ac{SISR}
	problem (scale = 4, dataset = SET14 and SET5).}
  \label{fig:psnr_complexty_sisr}
\end{figure}

Several conclusion can be drawn from \myfigref{fig:psnr_complexty_sisr}, which
could be confirmed also for the other validation datasets used:

\begin{itemize}
\item The model \textit{no\_tad} is the baseline model which is trained to
only upscale based on the guidance image, i.e. the standard approach to solve
the \ac{SISR} problem. Although the reconstruction performance of the
\textit{no\_tad} model is worse than state-of-the-art methods (e.g. VDSR
\cite{AISRUVDCN} shown in \myfigref{fig:vdsr}), clearly the task-aware
downscaling approach could largely improve the performance, compared the equivalent
\textit{aetad} model which is trained using task aware downscaling. In fact
for similar performance more than about $25 \%$ of the model parameters can
be omitted, still resulting in a better accuracy (comparison \textit{no\_tad}
vs \textit{aetad\_small}).
\item The \textit{Resblocks} have a large impact on the models performance,
purely convolutional models with neither a skip connection nor a ReLU layer
(as occurring in \textit{Resblocks}), such as the models \textit{conv\_only} or
\textit{conv\_only\_very\_large}, have an overall worse reconstruction
performance with a similar number of parameters.
\item In a direct comparison the model \textit{aetad\_direct4} performs worse
than the iteratively scaling structured, but otherwise equivalent model
\textit{aetad}.
\item In both displayed validation datasets the reconstruction performance
stagnates with increasing number of parameters, as the difference in \ac{PSNR}
between the \textit{aetad\_large} and the \textit{aetad\_very\_large}
model does not improve much anymore.
\end{itemize}

While more complex architectures than the baseline (\textit{Kim et al.}-model)
do not gain a lot of accuracy, for less complex models with similar architecture
there is a large drop in accurcy. Therefore, the baseline architecture already
is very reasonable.

\myfigref{fig:sisr_different_scales} shows the \ac{PSNR} curves for a model
trained on a scaling factor of 2, while being validated on both of the scaling
factor 2 and 4.

\begin{figure}[!htbp]
	\centering
	\includegraphics[width=10cm]{figures/sisr_different_scales}
	\caption{\ac{PSNR} curve on SET5 validation dataset for different scales
	(trained scaling factor = 2).}
  \label{fig:sisr_different_scales}
\end{figure}

\begin{table}[!htbp]
	\begin{center}
	\begin{tabular}{c|c|c|c|c}
	model & x2 & x4 & x8 & x16 \\
	\hline
  \textit{no\_tad} & 7.457 & 27.204 & 7.459 & 19.488 \\
	\textit{aetad\_very\_small} & 8.527 & 28.334 & 7.780 & 18.799 \\
	\textit{aetad\_very\_large} & 5.882 & 29.617 & 5.928 & 18.673 \\
	\end{tabular}
	\caption{Comparison of the reconstruction performance on non-trained scales
	between task aware and non task aware trained models on SET14.}
	\label{table:sisrotherscales}
	\end{center}
\end{table}


\begin{table}[!htbp]
	\begin{center}
	\begin{tabular}{c|c|c|c|c}
	scale & dataset & PSNR (Kim et al.) & PSNR (\textit{aetad\_skip2})
	& PSNR (\textit{aetad\_very\_small}) \\
	\hline
  x4 & SET5 & 31.81 & 31.814 & 30.302 \\
	x4 & SET14 & 28.63 & 28.665 & 28.334 \\
	x4 & URBAN100 & 26.63 & 24.156 & 23.084 \\
	x4 & BSDS100 & 28.51 & 28.601 & 25.719 \\
	\end{tabular}
	\caption{Comparison of reconstruction accuracy between Kim et al. \cite{TAID},
	\textit{aetad\_skip2} and \textit{aetad\_very\_small}
	model for scaling factor 4 on several validation datasets. }
	\label{table:sisperformance}
	\end{center}
\end{table}

\subsection{Image Colorization}
\label{sec:Experiments_IC}


\begin{figure}[!htbp]
	\centering
	\includegraphics[width=18cm]{figures/psnr_complexity_ic}
	\caption{Model complexity vs reconstruction performance for \ac{IC}
	problem (dataset = SET14 and SET5).}
  \label{fig:psnr_complexity_ic}
\end{figure}

\begin{table}[!htbp]
	\begin{center}
	\begin{tabular}{c|c|c|c|c}
	scale & dataset & PSNR (Kim et al.)
	& PSNR (\textit{aetad\_color\_large}) & $\frac{\# params kim et al.}{\# params aetad\_color\_large}$  \\
	\hline
  x4 & SET5 & None & 34.416 \\
	x4 & SET14 & None & 31.262 \\
	x4 & URBAN100 & 33.68 & 33.604 \\
	x4 & BSDS100 & 36.14 & 36.786 \\
	\end{tabular}
	\caption{Comparison of reconstruction accuracy between Kim et al. \cite{TAID},
	\textit{aetad\_color\_large} model on several validation datasets. }
	\label{table:icperformance}
	\end{center}
\end{table}

\subsection{Video Super-Resolution}
\label{sec:Experiments_VSR}

\begin{table}[!htbp]
	\begin{center}
	\begin{tabular}{c|c|c|c|c}
	scale & dataset & \ac{SISR} model & non task aware SOFVSR
	& task aware SOFVSR \\
	\hline
  x4 & CALENDAR & 21.297 & 19.190 & 18.573 \\
	x4 & CITY & 25.332 & 24.677 & 24.191 \\
	\end{tabular}
	\caption{Comparison of reconstruction accuracy of \ac{SISR} model and
	the rebuild SOFVSR model without and with \ac{TAD}.}
	\label{table:vsrperformance}
	\end{center}
\end{table}


\begin{enumerate}
\item super-large scale for videos
\end{enumerate}

\subsection{Qualitative Improvements}
\label{sec:Experiments_QI}
As demonstrated above \ac{TAD} is able to improve the performance of
image reconstruction models quantitatively, but the \ac{TAD} approach also
improves the results in a qualitative manner, in sense of that images can be
restored that would be able to be restored from the trivially downscaled image,
in the following shown using the example of a \ac{IC} task.
\newline
Consider \myfigref{fig:gummibears} displaying objects with equivalent shape
but different colors (gummibears). While by using trivial downscaling methods
like averaging over the colors (grayscale) some color information are
unrecoverably lost, e.g. the yellow and orange gummibear looking nearly
equivalent in grayscale, a task aware approach learns to keep basic color
information despite of downscaling, figuratively speaking. Hence, even the color
of shapes which are (exactly) similar in grayscale can be restored.

\begin{figure}[!htbp]
	\centering
	\includegraphics[width=8cm]{figures/gummibears_GRY}
	\includegraphics[width=8cm]{figures/gummibears_COL}
	\includegraphics[width=8cm]{figures/gummibears_SCOLT_notad}
	\includegraphics[width=8cm]{figures/gummibears_SCOLT}
	\caption{Image colorization of similar shapes (from upper left to lower
	right: grayscale, groundtruth, colorized without \ac{TAD}, colorized with \ac{TAD})}
  \label{fig:gummibears}
\end{figure}

% Describe the evaluation you did in a way, such that an independent researcher can repeat it. Cover the following questions:
% \begin{itemize}
%  \item \textit{What is the experimental setup and methodology?} Describe the setting of the experiments and give all the parameters in detail which you have used. Give a detailed account of how the experiment was conducted.
%  \item \textit{What are your results?} In this section, a \emph{clear description} of the results is given. If you produced lots of data, include only representative data here and put all results into the appendix.
% \end{itemize}

\newpage
\chapter{Discussion and Conclusion}
\label{sec:Discussion}

\begin{wrapfigure}{r}{6cm}
    \centering
    \includegraphics[width=6cm]{figures/aetad_very_small_runtime.png}
    \caption{Runtime on Mac Pro 2015 for one scaling operation using \textit{aetad\_very\_small} in comparison to 37s (x2) / 51s (x4) using baseline model.}
    \label{fig:aetad_very_small_runtime}
\end{wrapfigure}

The goals of this work were to improve the performance-runtime tradeoff of task aware downscaling for the single-image super-resolution problem and image colorization problem, to increase the robustness against perturbations and to extend it to the video domain, shown based on the video super-resolution task.
\newline
As largely presented in \mysecref{sec:ExperimentsandResults} more reasonable models could be found by adaptions of both the model architecture and the training procedure. Thus, for the \ac{SISR} problem a model could be derived with slightly worse (but mostly similar) performance but
$44 \%$ less model parameters, compared to the baseline architecture by Kim et al. \cite{TAID}, which highly decreases the model's runtime on non-GPU hardware (from $51s$ to about $18s$ for a single forward pass with scaling factor $4$ on a Mac Pro 2015, comp. \myfigref{fig:aetad_very_small_runtime}). Also for the \ac{IC} task a more reasonable model was derived. Due to the reconstruction performance peaking with a slight increase of the model's complexity, the final model is larger, but increases reconstruction accuracy by $31 \%$. Likewise a method to improve robustness against perturbation and performance ($\epsilon = 1$) as well as the trade-off between reconstruction performance with and without perturbation was demonstrated. As shown introducing \ac{TAD} to the video domain by training an external \ac{VSR} model (here Wang et al. \cite{LFVSRTHROFE}) improves its reconstruction accuracy on every validation dataset used. Overall the \ac{TAD} purposed by Kim et al. was thus widely improved and extended within this work. 
\newline 
In general \ac{TAD} can improve the reconstruction performance both quantitatively in terms of \ac{PSNR} as well as qualitatively in terms of enabling reconstruction of otherwise irreconstrucable images, while being very efficient compared to other state-of-the-art models (in terms of model complexity, e.g. compared to \myfigref{fig:vdsr} having 20 layers for upscaling only). However, it comes at the price of computational cost, as it is clearly more expensive than trivial downscaling methods such as averaging over channels or bilinear/bicubic interpolation. Nevertheless this work contributed to a more efficient way of task aware downscaling there still is some work to do in order to further compress the models used to make them practicable on non-GPU devices such as mobile phones, so that it can be used as alternative of bilinear compression e.g. in messenger or mailing services. Also the underlying project demonstrated the feasibility of applying \ac{TAD} on the video domain for the \ac{VSR} problem combined with a scaling factor up to $4$. Further work could integrate other external models as well as use examine the performance for larger scaling factors such as $16$. Furthermore during this work it was assumed to have a constant compression rate while improving the reconstruction performance. Although the model already outperforms standard compression algorithms such as jpeg further work can be done to meet both the readability constraint of the low-dimensional image as well as a more dense representation of the high-dimensional data. The \ac{TAD} architecture used within is very much constrained on the size on the size of the guidance image, although there might be a more dense representation of the high-dimensional data (due to the readability constraint). However, a double autoencoder structure as displayed in \myfigref{fig:future_work_double_tad} could resolve the problem, by first downscale to a human understandable representation and then further compress to a more dense representation. Thereby, to avoid interdependent optimization therefore the outer autoencoder could be fixed while training the inner autoencoder. 

\begin{figure}[!htbp]
	\centering
	\includegraphics[width=16cm]{figures/future_work_double_tad.png}
	\caption{Double \ac{TAD} architecture for improvement in compression rate.}
  \label{fig:future_work_double_tad}
\end{figure}

% The discussion section gives an interpretation of what you have done :
%
% \begin{itemize}
%  \item \textit{What do your results mean?} Here you discuss, but you do not recapitulate results. Describe principles, relationships and generalizations shown. Also, mention inconsistencies or exceptions you found.
%  \item \textit{How do your results relate to other's work?} Show how your work agrees or disagrees with other's work. Here you can rely on the information you presented in the ``related work'' section.
%  \item \textit{What are implications and applications of your work?} State how your methods may be applied and what implications might be.
% \end{itemize}
%
% \noindent Make sure that introduction/related work and the discussion section act as a pair, i.e. ``be sure the discussion section answers what the introduction section asked'' .

% List the conclusions of your work and give evidence for these.



%% ----------------------------------------------------------------------------
% If Appendix is needed
%% ----------------------------------------------------------------------------
\appendix
\newpage
\section{Appendix}
\label{sec:Appendix}

\subsection*{Video-Super-Resolution using Optical Flow Reconstruction}
One approach that was tried in order to apply \ac{TAD} on \ac{VSR} was to find a
low-dimensional representation that is optimal to reconstruct the optical flow.
\myfigref{fig:architecture_video_flow} indicates the design for this
approach, which was trained using the UCL optical flow dataset (\cite{ADAEMFOF})
and the OpenCV Farneback dense optical flow function (since easily available
for a conceptual test).
Unfortunately, the approach did not converge to useful representation. Since
other approaches could successfully solve the \ac{VSR} task this approach was
discarded.

\begin{figure}[!htbp]
	\centering
	\includegraphics[width=14cm]{figures/architecture_video_flow}
	\caption{Design for \ac{TAD} video super-resolution task (example network
  architecture) for optical flow reconstruction. }
  \label{fig:architecture_video_flow}
\end{figure}

\begin{figure}[!htbp]
	\centering
	\includegraphics[width=10cm]{figures/flow}
	\caption{Optical flow reconstruction: Groundtruth (left) and reconstructed
  image (right).}
  \label{fig:flow}
\end{figure}

\subsection*{SOFVSR model selection and working principle}
Wang et al. \cite{LFVSRTHROFE} (SOFVSR) implemented an end-to-end trainable
approach to predict both, the \ac{HR} frame as well as the HR optical flow.
Therefore, first the HR optical flow is inferred in a coarse-to-fine manner,
then motion compensation is performed according to the HR optical flows and
finally, the compensated LR inputs are fed to a super-resolution network to
generate the HR frame estimate (comp. \myfigref{fig:sofvsr}).

\begin{figure}[!htbp]
	\centering
	\includegraphics[width=14cm]{figures/sofvsr}
	\caption{Overview of SOFVSR pipeline \cite{LFVSRTHROFE}.}
  \label{fig:sofvsr}
\end{figure}

The SOFVSR model was selected based on a combination of criteria such as the
stated reconstruction performance (PSNR) and runtime, the closeness to
state-of-the-art and the availability of a PyTorch open-source implementation.

\begin{figure}[!htbp]
	\centering
	\includegraphics[width=14cm]{figures/sofvsr_selection}
	\caption{Comparison of several \ac{VSR} methods according to PapersWithCode.}
  \label{fig:sofvsr_selection}
\end{figure}


% In the appendix, list the following material:
%
% \begin{itemize}
%  \item Data (evaluation tables, graphs etc.)
%  \item Program code
%  \item Further material
% \end{itemize}


%% ----------------------------------------------------------------------------
% Bibliography is stored in references.bib.
%% ----------------------------------------------------------------------------
\bibliographystyle{plain}
\addcontentsline{toc}{section}{References}
\bibliography{references}

\end{document}
